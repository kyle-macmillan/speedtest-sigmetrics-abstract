\begin{abstract}
	Consumers, regulators, and ISPs all use client-based ``speed tests'' to
	measure network performance, both in single-user settings and in aggregate.
	Two prevalent speed tests, Ookla's Speedtest and Measurement Lab's Network
	Diagnostic Test (NDT), are often used for similar purposes, despite having
	significant differences in both the test design and implementation, and in
	the infrastructure used to perform measurements.  In this paper, we present
	the first-ever comparative evaluation of \ookla and \ndt (the latest version
	of NDT), both in controlled and wide-area settings.  Our goal is to
	characterize when and to what extent these two speed tests yield different
	results, as well as the factors that contribute to the differences. To study
	the effects of the test design, we conduct a series of controlled, in-lab
	experiments under a comprehensive set of network conditions and usage modes
	(e.g., TCP congestion control, native vs. browser client). Our results show
	that \ookla and \ndt report similar speeds under most in-lab conditions,
	with the exception of networks that experience high latency, where \ookla
	consistently reports higher throughput. To characterize the behavior of
	these tools in wide-area deployment, we collect more than 80,000 pairs of
	\ookla and \ndt measurements across nine months and 126 households, with a
	range of ISPs and speed tiers. This first-of-its-kind paired-test analysis
	reveals many previously unknown systemic issues, including high variability
	in \ndt test results and systematically under-performing servers in the
	\ookla network.
\end{abstract}

\section{Main Findings} \label{sec:findings}

Our main findings are summarized below:
\begin{itemize}
    \item   The \ndt client can send at about 95\% of a high-capacity link (up to
			2~Gbps) using only a single TCP connection. This finding updates past work
			that reported a different finding, that a single TCP connection can not achieve a
			throughput approaching full capacity~\cite{feamster2020measuring}. 
			(\cref{subsec:network-conditions}).\\\\
			
	\item	The \ndt client under-reports throughput at higher latencies, in comparison
			to \ookla: The \ookla client reports speeds up to 12\% higher than \ndt at
			200~ms round-trip latency, and up to 56\% higher at 500~ms latency
			(\cref{subsec:network-conditions}).\\\\
			
    \item	Across all households in the wide-area deployment, the median fraction of paired tests 
			for which \ookla
			reports a speed that is 0--5\% higher than \ndt is 73.8\%. The fraction of
			paired tests for which \ookla reports a speed that is 5--25\% higher is
			13.4\% (\cref{subsec:paired-results}).\\\\
			
	\item 	For \ookla, the choice of test server can significantly affect the reported
			speed. Tests using certain \ookla servers systematically report speeds 10\%
			lower than other servers. (\cref{subsec:server-selection}). \\\\
			
	\item	\ndt tests are more likely to under-report during peak hours. 43.4\% of
			households observed a statistically significant decrease in \ndt-reported
			download speed tests during peak hours, whereas only 18.9\% of these
			households saw the same for \ookla (\cref{subsec:time-of-day}).\\ \hline
\end{itemize}



\section{Conclusion}\label{sec:conclusion}

This work provides an in-depth comparison of \ookla and \ndt, focusing on both
test design and infrastructure~\cite{macmillan2023comparative}. Our results and
suggestions help users understand why the reported speed from Ookla and NDT7 may
differ, as well as guide policymakers towards more accurate and appropriate use
of \ookla and \ndt data.  We have released all collected data, as well as all of
the measurement code we used to conduct the
study~\cite{netrics-code,netrics-data}. 

Although this is the {\em first} work to perform a controlled and extensive
comparison of \ookla and \ndt, we neither hope nor expect that it will be the
last, as many important technical and policy questions still remain. In the
paper, we the implications of our findings for the future of speed test tools
and data analysis and outline multiple avenues for future work. We view this
research as the beginning of a discussion on how to use collective speed test
data to shed more light on the state of broadband Internet access networks
around the United States, and the world.

\noindent \textbf{Acknowledgments.} This work was supported by National Science
Foundation awards CNS-2224687 and CNS-2223610 and a {\tt data.org} Inclusive
Growth and Recovery Challenge Award.  We thank the reviewers and our shepherd,
Zubair Shafiq, for helpful comments. We are also grateful to Guilherme Martins,
Marc Richardson, and Grace Chu for their help in building the data collection
platform and recruiting participants for the study of residential networks.


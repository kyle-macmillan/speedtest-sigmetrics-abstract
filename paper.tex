\documentclass[sigconf]{acmart}


\usepackage{pdflscape}
\usepackage{comment}
\usepackage{xcolor}
\usepackage{float}
\usepackage{balance}
\usepackage{xspace}
\usepackage{subcaption}
\usepackage{multirow}
\usepackage{placeins}
\usepackage{placeins}
\usepackage{cleveref}
\hypersetup{citecolor = blue}
\usepackage[framemethod=TikZ]{mdframed}
\usepackage[utf8]{inputenc}
\graphicspath{{figures/}}
\usepackage[font=small,format=plain,labelfont=bf,textfont=it]{caption}
\usepackage{booktabs}\usepackage[all=normal,wordspacing]{savetrees}
\usepackage{enumitem}
%%
%% \BibTeX command to typeset BibTeX logo in the docs
\AtBeginDocument{%
  \providecommand\BibTeX{{%
    Bib\TeX}}}

%% Rights management information.  This information is sent to you
%% when you complete the rights form.  These commands have SAMPLE
%% values in them; it is your responsibility as an author to replace
%% the commands and values with those provided to you when you
%% complete the rights form.
\copyrightyear{2023}
\acmYear{2023}
\setcopyright{rightsretained}
\acmConference[SIGMETRICS '23 Abstracts] {Abstract Proceedings of the 2023 ACM SIGMETRICS International Conference on Measurement and Modeling of Computer Systems}{June 19--23, 2023}{Orlando, FL, USA.}
\acmBooktitle{Abstract Proceedings of the 2023 ACM SIGMETRICS International Conference on Measurement and Modeling of Computer Systems (SIGMETRICS '23 Abstracts), June 19--23, 2023, Orlando, FL, USA}
\acmPrice{}
\acmISBN{979-8-4007-0074-3/23/06}
\acmDOI{10.1145/3579448}


\newcommand{\ookla}{Ookla\xspace}
\newcommand{\ndt}{NDT7\xspace}


\begin{document}
\begin{abstract}
	Consumers, regulators, and ISPs all use client-based ``speed tests'' to
	measure network performance, both in single-user settings and in aggregate.
	Two prevalent speed tests, Ookla's Speedtest and Measurement Lab's Network
	Diagnostic Test (NDT), are often used for similar purposes, despite having
	significant differences in both the test design and implementation, and in
	the infrastructure used to perform measurements.  In this paper, we present
	the first-ever comparative evaluation of \ookla and \ndt (the latest version
	of NDT), both in controlled and wide-area settings.  Our goal is to
	characterize when and to what extent these two speed tests yield different
	results, as well as the factors that contribute to the differences. To study
	the effects of the test design, we conduct a series of controlled, in-lab
	experiments under a comprehensive set of network conditions and usage modes
	(e.g., TCP congestion control, native vs. browser client). Our results show
	that \ookla and \ndt report similar speeds under most in-lab conditions,
	with the exception of networks that experience high latency, where \ookla
	consistently reports higher throughput. To characterize the behavior of
	these tools in wide-area deployment, we collect more than 80,000 pairs of
	\ookla and \ndt measurements across nine months and 126 households, with a
	range of ISPs and speed tiers. This first-of-its-kind paired-test analysis
	reveals many previously unknown systemic issues, including high variability
	in \ndt test results and systematically under-performing servers in the
	\ookla network.
}


%%
%% The "title" command has an optional parameter,
%% allowing the author to define a "short title" to be used in page headers.

\title{A Comparative Analysis of Ookla Speedtest and Measurement Labs Network
Diagnostic Test (NDT7)}

%%
%% The "author" command and its associated commands are used to define
%% the authors and their affiliations.
%% Of note is the shared affiliation of the first two authors, and the
%% "authornote" and "authornotemark" commands
%% used to denote shared contribution to the research.

\author{Kyle MacMillan}
\affiliation{%
  \institution{University of Chicago}
  \city{Chicago}
  \country{USA}}
\email{macmillan@uchicago.edu}

\author{Tarun Mangla}
\affiliation{%
  \institution{University of Chicago}
  \city{Chicago}
  \country{USA}}
\email{tmangla@uchicago.edu}

\author{James Saxon}
\affiliation{%
  \institution{University of Chicago}
  \city{Chicago}
  \country{USA}}
\email{jsaxon@uchicago.edu}

\author{Nicole P. Marwell}
\affiliation{%
  \institution{University of Chicago}
  \city{Chicago}
  \country{USA}}
\email{nmarwell@uchicago.edu}

\author{Nick Feamster}
\affiliation{%
  \institution{University of Chicago}
  \city{Chicago}
  \country{USA}}
\email{feamster@uchicago.edu}

%%
%% By default, the full list of authors will be used in the page
%% headers. Often, this list is too long, and will overlap
%% other information printed in the page headers. This command allows
%% the author to define a more concise list
%% of authors' names for this purpose.
\renewcommand{\shortauthors}{Kyle MacMillan et al.}

%%
%% The abstract is a short summary of the work to be presented in the
%% article.

%%
%% The code below is generated by the tool at http://dl.acm.org/ccs.cfm.
%% Please copy and paste the code instead of the example below.
%%

\begin{CCSXML}
<ccs2012>
   <concept>
	   <concept_id>10003033.10003079.10011704</concept_id>
	   <concept_desc>Networks~Network measurement</concept_desc>
	   <concept_significance>500</concept_significance>
	   </concept>
   <concept>
	   <concept_id>10003033.10003079.10011672</concept_id>
	   <concept_desc>Networks~Network performance analysis</concept_desc>
	   <concept_significance>500</concept_significance>
	   </concept>
 </ccs2012>
\end{CCSXML}

\ccsdesc[500]{Networks~Network measurement}
\ccsdesc[500]{Networks~Network performance analysis}

\keywords{Speed Test, Ookla, Network Diagnostic Tool, Broadband, Internet
Speed, Measurement Lab}

%%
%% This command processes the author and affiliation and title
%% information and builds the first part of the formatted document.
\maketitle

\section{Introduction}\label{sec:intro}

Network throughput---colloquially referred to as ``speed''---is among the most
well-established and widely used network performance metrics.  Indeed, ``speed''
is used as the basis for a wide range of purposes, from network troubleshooting
and diagnosis, to policy
advocacy~\cite{battle-for-the-net} (e.g., on
issues related to digital equity), to regulation and
litigation~\cite{fcc2022frontier} (e.g., on issues related to ISP advertised
speed).  Given the extent to which stakeholders, from consumers to regulators to
ISPs, all rely on ``speed'', it is in some sense surprising that there is no
consensus on the way to measure it. Absent any standard, many speed tests, varying in
both design and implementation, are used interchangeably. 

Over the past decade, Ookla's Speedtest~\cite{ookla2022speedtest} (``\ookla'')
and Measurement Lab's Network Diagnostic Tool (``NDT'')~\cite{mlab2022speedtest}
have been widely used by both consumers and policymakers: \ookla and NDT report
a daily average of over $10$ million and  $6$ million
tests, respectively. As a result, the compiled datasets
from these two tests, amounting to billions of speed
tests, have become universal
resources for analyzing broadband Internet performance~\cite{battle-for-the-net,
fcc2022frontier}. Unfortunately, these datasets have
also been used out of context, without a clear understanding of the caveats and
limitations of these tools under different circumstances and
environments~\cite{ookla-ny-case-study}. 

The stakes---and, therefore, the costs---of misuse have also never been higher. In the United
States, Congress has committed \$43.5 billion to Internet infrastructure,
including to last-mile performance and availability
improvements. 
% Allocating these funds depends, in
% part, on \ookla and NDT speed test data~\cite{ntia2018broadband}. 
In response,
state and local officials across the country are currently urging consumers to
participate in speed test crowd-sourcing initiatives to help establish which
areas meet the federal funding criteria.

To their credit, the organizations who have developed these speed test tools
have tried to prevent misappropriation of the data by issuing guidance about how
the tools and public data should and should not be used. M-Lab has gone as far
as to say that ``\textit{M-Lab's NDT and Ookla's SpeedTest measure fundamentally
different things}''~\cite{mlab-issue-email-1}. While this statement is certainly
true, there has been no study to date about how these differences in tool design
can (and do) yield different results in practice, under different operating
conditions. Acknowledging that \ookla and \ndt are different is, in some sense,
besides the point. %, for two
% reasons. 
Although each tool may have been designed with a specific purpose in mind, that
does not mean it can not fulfill---or be appropriated for---other purposes. Such
has been the case with \ndt, which has been used as a tool to measure access ISP
throughput, even though its stated design is to test the throughput of {\em a
single TCP connection}. In light of the significant attention to both of these
tests, it is imperative to develop a rigorous, quantitative, and specific
understanding of the circumstances under which each tool can accurately measure
last-mile speed---and, hence, the context for interpreting each dataset.

To this end, we conduct the first-of-its-kind systematic, comparative study of
\ndt (the latest version of NDT) and \ookla~\cite{macmillan2023comparative} \footnote{We focus on \ookla and \ndt because of   
their popularity with consumers and policy 
makers, but
the method in this paper also applies to other tools.}. We begin 
with a set of in-lab
experiments that allow us to directly compare the tools under controlled network
conditions where ``ground truth'' is known. Next, we conduct more than 80,000
paired wide-area network tests, whereby the two tests are run back-to-back, from
126 home broadband access networks across more than 30 neighborhoods in
one of the largest cities in the United States for nearly a year. In-lab, we use
controlled experiments to characterize how \ndt and \ookla behave under a wide
range of network conditions---specifically, varying throughput, latency, packet
loss, and cross-traffic. We also study how different transport congestion
control algorithms and client types (i.e., browser vs. native client) may affect
the measurements that each tool reports.  Second, we compare the behavior of
these two tools using data from our wide-area network deployment encompassing 10
different ISPs. A unique and important methodological aspect of our study is the
use of \textit{paired speed tests}, where we run \ookla and \ndt in succession.
To our knowledge, this is the first comparative analysis of \ookla and \ndt in
deployment over a significant number of networks for an extended period of time. 



\section{Main Findings} \label{sec:findings}

Our main findings are summarized below:
\begin{itemize}
    \item   The \ndt client can send at about 95\% of a high-capacity link (up to
			2~Gbps) using only a single TCP connection. This finding updates past work
			that reported a different finding, that a single TCP connection can not achieve a
			throughput approaching full capacity~\cite{feamster2020measuring}. 
			(\cref{subsec:network-conditions}).\\\\
			
	\item	The \ndt client under-reports throughput at higher latencies, in comparison
			to \ookla: The \ookla client reports speeds up to 12\% higher than \ndt at
			200~ms round-trip latency, and up to 56\% higher at 500~ms latency
			(\cref{subsec:network-conditions}).\\\\
			
    \item	Across all households in the wide-area deployment, the median fraction of paired tests 
			for which \ookla
			reports a speed that is 0--5\% higher than \ndt is 73.8\%. The fraction of
			paired tests for which \ookla reports a speed that is 5--25\% higher is
			13.4\% (\cref{subsec:paired-results}).\\\\
			
	\item 	For \ookla, the choice of test server can significantly affect the reported
			speed. Tests using certain \ookla servers systematically report speeds 10\%
			lower than other servers. (\cref{subsec:server-selection}). \\\\
			
	\item	\ndt tests are more likely to under-report during peak hours. 43.4\% of
			households observed a statistically significant decrease in \ndt-reported
			download speed tests during peak hours, whereas only 18.9\% of these
			households saw the same for \ookla (\cref{subsec:time-of-day}).\\ \hline
\end{itemize}



\section{Conclusion}\label{sec:conclusion}

This work provides an in-depth comparison of \ookla and \ndt, focusing on both
test design and infrastructure~\cite{macmillan2023comparative}. Our results and
suggestions help users understand why the reported speed from Ookla and NDT7 may
differ, as well as guide policymakers towards more accurate and appropriate use
of \ookla and \ndt data.  We have released all collected data, as well as all of
the measurement code we used to conduct the
study~\cite{netrics-code,netrics-data}. 

Although this is the {\em first} work to perform a controlled and extensive
comparison of \ookla and \ndt, we neither hope nor expect that it will be the
last, as many important technical and policy questions still remain. In the
paper, we the implications of our findings for the future of speed test tools
and data analysis and outline multiple avenues for future work. We view this
research as the beginning of a discussion on how to use collective speed test
data to shed more light on the state of broadband Internet access networks
around the United States, and the world.




%%
%% The next two lines define the bibliography style to be used, and
%% the bibliography file.
\bibliographystyle{ACM-Reference-Format}
\bibliography{paper}


%%
%% If your work has an appendix, this is the place to put it.

\end{document}
\endinput
%%
%% End of file `sample-sigconf.tex'.
